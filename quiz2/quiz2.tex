%%%%%%%%%%%%%%%%%%%%%%%%%%%%%%%%%%%%%%%%%
% Structured General Purpose Assignment
% LaTeX Template
%
% This template has been downloaded from:
% http://www.latextemplates.com
%
% Original author:
% Ted Pavlic (http://www.tedpavlic.com)
%
% Note:
% The \lipsum[#] commands throughout this template generate dummy text
% to fill the template out. These commands should all be removed when 
% writing assignment content.
%
%%%%%%%%%%%%%%%%%%%%%%%%%%%%%%%%%%%%%%%%%

%----------------------------------------------------------------------------------------
%	PACKAGES AND OTHER DOCUMENT CONFIGURATIONS
%----------------------------------------------------------------------------------------

\documentclass{article}

\usepackage{fancyhdr} % Required for custom headers
\usepackage{lastpage} % Required to determine the last page for the footer
\usepackage{extramarks} % Required for headers and footers
\usepackage{graphicx} % Required to insert images
\usepackage{lipsum} % Used for inserting dummy 'Lorem ipsum' text into the template
\usepackage{listings}
\usepackage{color}
\usepackage{amsmath}

\definecolor{dkgreen}{rgb}{0,0.6,0}
\definecolor{gray}{rgb}{0.5,0.5,0.5}
\definecolor{mauve}{rgb}{0.58,0,0.82}

\lstset{frame=tb,
  language=Python,
  aboveskip=3mm,
  belowskip=3mm,
  showstringspaces=false,
  columns=flexible,
  basicstyle={\small\ttfamily},
  numbers=none,
  numberstyle=\tiny\color{gray},
  keywordstyle=\color{blue},
  commentstyle=\color{dkgreen},
  stringstyle=\color{mauve},
  breaklines=true,
  breakatwhitespace=true
  tabsize=3
}

% Margins
\topmargin=-0.45in
\evensidemargin=0in
\oddsidemargin=0in
\textwidth=6.5in
\textheight=9.0in
\headsep=0.25in 

\linespread{1.1} % Line spacing

% Set up the header and footer
\pagestyle{fancy}
\lhead{\hmwkAuthorName} % Top left header
\chead{\hmwkClass\ (\hmwkClassInstructor\ \hmwkClassTime): \hmwkTitle} % Top center header
\rhead{\firstxmark} % Top right header
\lfoot{\lastxmark} % Bottom left footer
\cfoot{} % Bottom center footer
\rfoot{Page\ \thepage\ of\ \pageref{LastPage}} % Bottom right footer
\renewcommand\headrulewidth{0.4pt} % Size of the header rule
\renewcommand\footrulewidth{0.4pt} % Size of the footer rule

\setlength\parindent{0pt} % Removes all indentation from paragraphs

%----------------------------------------------------------------------------------------
%	DOCUMENT STRUCTURE COMMANDS
%	Skip this unless you know what you're doing
%----------------------------------------------------------------------------------------

% Header and footer for when a page split occurs within a problem environment
\newcommand{\enterProblemHeader}[1]{
\nobreak\extramarks{#1}{#1 continued on next page\ldots}\nobreak
\nobreak\extramarks{#1 (continued)}{#1 continued on next page\ldots}\nobreak
}

% Header and footer for when a page split occurs between problem environments
\newcommand{\exitProblemHeader}[1]{
\nobreak\extramarks{#1 (continued)}{#1 continued on next page\ldots}\nobreak
\nobreak\extramarks{#1}{}\nobreak
}

\setcounter{secnumdepth}{0} % Removes default section numbers
\newcounter{homeworkProblemCounter} % Creates a counter to keep track of the number of problems

\newcommand{\homeworkProblemName}{}
\newenvironment{homeworkProblem}[1][Problem \arabic{homeworkProblemCounter}]{ % Makes a new environment called homeworkProblem which takes 1 argument (custom name) but the default is "Problem #"
\stepcounter{homeworkProblemCounter} % Increase counter for number of problems
\renewcommand{\homeworkProblemName}{#1} % Assign \homeworkProblemName the name of the problem
\section{\homeworkProblemName} % Make a section in the document with the custom problem count
\enterProblemHeader{\homeworkProblemName} % Header and footer within the environment
}{
\exitProblemHeader{\homeworkProblemName} % Header and footer after the environment
}

\newcommand{\problemAnswer}[1]{ % Defines the problem answer command with the content as the only argument
\noindent\framebox[\columnwidth][c]{\begin{minipage}{0.98\columnwidth}#1\end{minipage}} % Makes the box around the problem answer and puts the content inside
}

\newcommand{\homeworkSectionName}{}
\newenvironment{homeworkSection}[1]{ % New environment for sections within homework problems, takes 1 argument - the name of the section
\renewcommand{\homeworkSectionName}{#1} % Assign \homeworkSectionName to the name of the section from the environment argument
\subsection{\homeworkSectionName} % Make a subsection with the custom name of the subsection
\enterProblemHeader{\homeworkProblemName\ [\homeworkSectionName]} % Header and footer within the environment
}{
\enterProblemHeader{\homeworkProblemName} % Header and footer after the environment
}
   
%----------------------------------------------------------------------------------------
%	NAME AND CLASS SECTION
%----------------------------------------------------------------------------------------

\newcommand{\hmwkTitle}{Quiz\ 2} % Assignment title
\newcommand{\hmwkDueDate}{July 16,\ 2014} % Due date
\newcommand{\hmwkClass}{Advanced Calculus with FE Application} % Course/class
\newcommand{\hmwkClassTime}{} % Class/lecture time
\newcommand{\hmwkClassInstructor}{Advisor: Dan Stefanica} % Teacher/lecturer
\newcommand{\hmwkAuthorName}{Weiyi Chen} % Your name

%----------------------------------------------------------------------------------------
%	TITLE PAGE
%----------------------------------------------------------------------------------------

\title{
\vspace{2in}
\textmd{\textbf{\hmwkClass:\ \hmwkTitle}}\\
\normalsize\vspace{0.1in}\small{Due\ on\ \hmwkDueDate}\\
\vspace{0.1in}\large{\textit{\hmwkClassInstructor\ \hmwkClassTime}}
\vspace{3in}
}

\author{\textbf{\hmwkAuthorName}}
\date{} % Insert date here if you want it to appear below your name

%----------------------------------------------------------------------------------------

\begin{document}

\maketitle

%----------------------------------------------------------------------------------------
%	TABLE OF CONTENTS
%----------------------------------------------------------------------------------------

%\setcounter{tocdepth}{1} % Uncomment this line if you don't want subsections listed in the ToC

%\newpage
%\tableofcontents
\newpage

%----------------------------------------------------------------------------------------
%	PROBLEM 1
%----------------------------------------------------------------------------------------

\begin{homeworkProblem}
    The approximate values of the integral
    \begin{equation}
         \frac{1}{2} + \frac{1}{\sqrt{2\pi}}\int_0^t e^{-\frac{x^2}{2}}dx
    \end{equation} 
    using the Simpson's rules can be found in the table below:
    \begin{table}[h] \centering
        \begin{tabular}{l|lll}
No. Intervals & N(0.1)         & N(0.5)         & N(1.0)         \\ \hline
4             & 0.539827837293 & 0.691462502398 & 0.841345406139 \\
8             & 0.539827837278 & 0.69146246384  & 0.84134478715  \\
16            & 0.539827837277 & 0.691462461434 & 0.841344748633 \\
32            &                & 0.691462461284 & 0.841344746229 \\
64            &                & 0.691462461275 & 0.841344746079 \\
128           &                & 0.691462461274 & 0.841344746069 \\
256           &                &                & 0.841344746069
        \end{tabular}
    \end{table} \\
    The approximate values of the integral are
    \begin{equation}
        N(0.1)=0.539827837277, N(0.5)=0.691462461274, N(1.0)=0.841344746069
    \end{equation}
    and are obtained for a 16 intervals partition, 128 intervals partition and 256 intervals partition, respectively, using Simpson's rule. Following is the python code as of the routine to compute above values
    \begin{lstlisting}
from math import *

def simpson(f, d_a, d_b, i_n):
    i_n *= 2
    d_h = (d_b - d_a) / i_n
    d_k, d_x = 0.0, d_a + d_h
    for i in range(1, i_n/2 + 1):
        d_k += 4*f(d_x)
        d_x += 2*d_h
    d_x = d_a + 2*d_h
    for i in range(1, i_n/2):
        d_k += 2*f(d_x)
        d_x += 2*d_h
    return (d_h/3)*(f(d_a)+f(d_b)+d_k)

def function(x): return exp(-x*x/2)

d_previous, d_current = 0.0, 0.0
for j in [0.1, 0.5, 1.0]:
    print "Compute N(", j, "):"
    for i in range(2,200):
        i_n = 2**i
        d_current = 0.5 + simpson(function, 0.0, j, i_n) / sqrt(2*pi)
        if abs(d_current-d_previous) < 10**(-12):
          print i_n, d_current
          break
        else:
          print i_n, d_current
          d_previous = d_current
    \end{lstlisting}
\end{homeworkProblem}
\newpage

%----------------------------------------------------------------------------------------
%   PROBLEM 2
%----------------------------------------------------------------------------------------

\begin{homeworkProblem}
    \begin{homeworkSection}{Annual coupon bond}
        If the bond is an annual coupon bond, the value of a 19 months bond with coupon rate 4\% and face value \$100 will be 
        \begin{equation}
            \begin{split}
                B &= \sum_{t=7/12} 100Ce^{-tr(0,t)} + (100C+100)e^{-Tr(0,T)} \\
                &= 4e^{-7/12r(0,7/12)} + 104e^{-19/12r(0, 19/12)}
            \end{split}
        \end{equation}
        The data below refers to the pseudocode from Table 2.5 for computing the bond price given the zero rate curve \\
        Input: $n=2$
        \begin{equation}
        \begin{split}
            t\_cash\_flow &= [7./12, 19./12] \\
            v\_cash\_flow &= [4., 104.]
        \end{split}
        \end{equation}
        The price of the bond is $B = 103.440082$.
    \end{homeworkSection}
    \begin{homeworkSection}{Semiannual coupon bond}
        Input: $n=4$
        \begin{equation}
        \begin{split}
            t\_cash\_flow &= [1./12, 7./12, 13./12, 19./12] \\
            v\_cash\_flow &= [2., 2., 2., 102.]   
        \end{split}      
        \end{equation}
        The price of the bond is $B = 103.495539$.
    \end{homeworkSection}
    \begin{homeworkSection}{Quarterly coupon bond}
        Input: $n=7$
        \begin{equation}
        \begin{split}
            t\_cash\_flow &= [1./12, 4./12,..., 16./12, 19./12] \\
            v\_cash\_flow &= [1., 1., ..., 101.]
        \end{split}        
        \end{equation}
        The price of the bond is $B = 102.518910$.
    \end{homeworkSection}
    \\Attached is the python code to compute quarterly coupon bond as an example -
    \begin{lstlisting}
from math import *

def r_2(time):
    return 0.02+time/(200.-time)

ls_cashflow = [1.,1.,1.,1.,1.,1.,101.]
ls_time = [1./12,4./12,7./12,10./12,13./12,16./12,19./12]
f_ret = 0.
for i in range(len(ls_cashflow)):
    f_ret += ls_cashflow[i]*exp(-ls_time[i]*r_2(ls_time[i]))
print f_ret
    \end{lstlisting}
\end{homeworkProblem}

%----------------------------------------------------------------------------------------
%   PROBLEM 3
%----------------------------------------------------------------------------------------

\newpage
\begin{homeworkProblem}
    The price, duration, and convexity of the bond can be obtained from the yield y of the bond as follows:
    \begin{equation}
        \begin{split}
            B &= \sum_{t=[1/12, 7/12, 13/12]} 2\exp(-yt) + 102\exp(-yT) \\
            D &= \frac{1}{B} (\sum_{t=[1/12, 7/12, 13/12]} 2t\exp(-yt) + 102T\exp(-yT)) \\
            C &= \frac{1}{B} (\sum_{t=[1/12, 7/12, 13/12]} 2t^2\exp(-yt) + 102T^2\exp(-yT)
        \end{split}
    \end{equation}
    where $T = 19/12$. \\
    The data below refers to the pseudocode from Table 2.7 for computing the price, duration and convexity of a bond given the yield of the bond. \\
    Input: $n = 4, y=2.5\%$
    \begin{equation}
        \begin{split}
            t\_cash\_flow &= [1./12, 7./12, 13./12, 19./12] \\
            v\_cash\_flow &= [2., 2., 2., 102.]
        \end{split}
    \end{equation}
    Output: bond price $B = 103.954808$, bond duration $D = 1.526212$, and bond convexity $C = 2.392899$. \\
    Attached is the python code to compute bond price, duration and convexity -
    \begin{lstlisting}
import numpy as np
from math import *

def bond_price(ls_cashflow, ls_time, f_yield):
    f_ret = 0.
    for i in range(len(ls_cashflow)):
        f_ret += ls_cashflow[i]*exp(-ls_time[i]*f_yield)
    return f_ret

def bond_duration(ls_cashflow, ls_time, f_yield):
    f_ret = 0.
    for i in range(len(ls_cashflow)):
        f_ret += ls_cashflow[i]*ls_time[i]*exp(-ls_time[i]*f_yield)
    return f_ret/bond_price(ls_cashflow, ls_time, f_yield)

def bond_convexity(ls_cashflow, ls_time, f_yield):
    f_ret = 0.
    for i in range(len(ls_cashflow)):
        f_ret += ls_cashflow[i]*ls_time[i]*ls_time[i]*exp(-ls_time[i]*f_yield)
    return f_ret/bond_price(ls_cashflow, ls_time, f_yield)

ls_cashflow = [2.,2.,2.,102]
ls_time = [1./12, 7./12, 13./12, 19./12]
f_yield = 0.025
print bond_price(ls_cashflow, ls_time, f_yield)
print bond_duration(ls_cashflow, ls_time, f_yield)
print bond_convexity(ls_cashflow, ls_time, f_yield)
    \end{lstlisting}
\end{homeworkProblem}

%----------------------------------------------------------------------------------------
%   PROBLEM 4
%----------------------------------------------------------------------------------------

\newpage
\begin{homeworkProblem}
    \begin{homeworkSection}{(i)}
        The value of the fixed leg of a 30 months semiannual swap with a \$10 million notional with fixed rate 3\% is
        \begin{equation}
            \begin{split}
                v_{fixed} &= (\sum_{t=[.5,1,1.5,2]} 10\times0.03/2[1+0.5r(0,t)]^{-t/0.5}) + 10\times(1+0.03/2)(1+0.5r(0,T))^{-T/0.5} \\
                &= (\sum_{t=[.5,1,1.5,2]} 0.15[1+0.5r(0,t)]^{-t/0.5}) + 10.15(1+0.5r(0,T))^{-T/0.5}
            \end{split}
        \end{equation}
        where $T = 2.5$. \\
        The data below refers to the pseudocode from Table 2.5 for computing the bond price given the zero rate curve \\
        Input: $n=5$
        \begin{equation}
            \begin{split}
                t\_cash\_flow &= [.5,1,1.5,2,2.5] \\
                v\_cash\_flow &= [0.15,0.15,0.15,0.15,10.15]
            \end{split}
        \end{equation}
        Output: the value is
        \begin{equation}
            v_{fixed} = 9.92141527551
        \end{equation}
        Therefore the value of the swap is 
        \begin{equation}
            v_{swap} = v_{float} - v_{fixed} = 10. - 9.92141527551 = 0.0785847244
        \end{equation}
        million dollars, that is $78,584.72$ dollars.
    \end{homeworkSection}
\newpage
    \begin{homeworkSection}{(ii)}
        The next floating payment happens at 1 month later as of \$125,000. Then the value of the floating leg at $t = 1/12$ is the amount of notional adding the next floating payment, that is
        \begin{equation}
            v_{float}(t = \frac{1}{12}) = 0.125 + 10 = 10.125
        \end{equation}
        Then the current value of the floating leg is
        \begin{equation}
            v_{float}(t=0) = v_{swap}(t) \times (1+0.5r(0,t))^{-t/0.5} = \$10.1038276352
        \end{equation}
        where $t = 1/12$. \\
        Now we calculate the fixed leg as the way in (i), that is \\
        Input: $n=5$
        \begin{equation}
            \begin{split}
                t\_cash\_flow &= [1./12,7./12,13./12,19./12,25./12] \\
                v\_cash\_flow &= [0.15,0.15,0.15,0.15,10.15]
            \end{split}
        \end{equation}
        Output: the value of the fixed leg is
        \begin{equation}
            v_{fixed} = 10.0863823028
        \end{equation}
        Therefore the value of the swap is 
        \begin{equation}
            v_{swap} = v_{float} - v_{fixed} = 0.0174453324728
        \end{equation}
        million dollars, that is $17,445.33$ dollars.
    \end{homeworkSection}
\end{homeworkProblem}


\end{document}